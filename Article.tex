%%%%%%%%%%%%%%%%%%%%%%%%%%%%%%%%%%%%%%%%%
% Article EcoFoG
% Version 2.3 (09/01/2021)
%
% adapté de :
% Stylish Article
% LaTeX Template
% Version 1.0 (31/1/13)
%
% This template has been downloaded from:
% http://www.LaTeXTemplates.com
%
% Original author:
% Mathias Legrand (legrand.mathias@gmail.com)
%
% License:
% CC BY-NC-SA 3.0 (http://creativecommons.org/licenses/by-nc-sa/3.0/)
%
%%%%%%%%%%%%%%%%%%%%%%%%%%%%%%%%%%%%%%%%%


%----------------------------------------------------------------------------------------
%	PACKAGES AND OTHER DOCUMENT CONFIGURATIONS
%----------------------------------------------------------------------------------------

\documentclass[fleqn,10pt]{ArtEcoFoG} % Document font size and equations flushed left


\usepackage{lipsum}      % Ajouter ici les packages nécessaires à la place de lipsum.

\setcounter{tocdepth}{3} % Niveau de détail de la table des matières.


%----------------------------------------------------------------------------------------
%	ARTICLE INFORMATION
%----------------------------------------------------------------------------------------

\JournalInfo{Hal xxxx} % Journal information
\Archive{DOI xxxx} % Additional notes (e.g. copyright, DOI, review/research article)

\PaperTitle{Titre de l'article} % Article title

\Authors{
  Prénom Nom\textsuperscript{1*}\\ Deuxième Auteur\textsuperscript{2}
} % Authors
\affiliation{
  \textsuperscript{1}UMR EcoFoG, AgroParistech, CNRS, Cirad, INRA, Université des Antilles,
  Université de Guyane.\\ \hspace{1em} Campus Agronomique, 97310 Kourou, France.\\
  \textsuperscript{2}Department of Ecology, University of Edimburgh\\ \hspace{1em} Street address, Zip code, Country.
}
\affiliation{*\textbf{Contact}: prenom.nom@ecofog.gf, http://www.ecofog.gf/spip.php?article47} % Corresponding author

\Keywords{mot-clés, séparés par des virgules} % Keywords - if you don't want any simply remove all the text between the curly brackets
\newcommand{\keywordname}{Mots-clés}          % (or Keywords) Defines the keywords heading name

%----------------------------------------------------------------------------------------
%	ABSTRACT
%----------------------------------------------------------------------------------------

\Abstract{
  Résumé de l'article.
}

%----------------------------------------------------------------------------------------

\begin{document}
  
\selectlanguage{french} % french ou english

\flushbottom            % Makes all text pages the same height
\maketitle              % Print the title and abstract box
\tableofcontents        % Print the contents section
\thispagestyle{empty}   % Removes page numbering from the first page

%----------------------------------------------------------------------------------------
%	ARTICLE CONTENTS
%----------------------------------------------------------------------------------------


\section{Introduction}\label{introduction}

Ce modèle permet la rédaction d'articles au format \LaTeX en Français ou en Anglais, avec un formatage correct.
Il produit des articles bien formatés pour l'auto-harchivage (dépôt sur HAL par exemple).

\section{Mode d’emploi}\label{sec:modemploi}

\subsection{Préambule}\label{sec:preambule}

Le titre de l'article, les noms des auteurs, le résumé, etc. sont saisis dans le préambule du document.

\subsection{Texte}\label{sec:texte}

Utiliser la syntaxe \LaTeX classique.

\subsection{Bibliographie}\label{sec:bibliographie}

Les références bibliographiques doivent être placées dans le fichier \code{references.bib}.
Elles sont appelées par les commandes \code{\textbackslash citep} \citep{Marcon2014c} ou  \code{\textbackslash cite}, comme \cite{Marcon2014c}.


\section{Mise en page}\label{sec:misenpage}

Les figures et tableaux occupent une colonne.
Ajouter une étoile après la commande pour une mise en page en pleine largeur; voir par exemple le tableau~\ref{table:step1}.

\begin{table*}
  \caption{Estimating the number of neighbors}
  \centering
  \begin{tabularx}{\textwidth}{p{3cm} p{7cm} X}
    \toprule
    Function & Neighbors around $x_i$ & Observations \\
    \midrule
    
    $\widehat{K} \left( r \right) / \left( \pi r^2 \right)$ 
    & $\displaystyle n\left(x_i,r\right) = \sum_{j, i \ne j} \mathbf{1} \left(\left\|x_i - x_j \right\| \le r \right) c\left( i,j \right)$
    & The number of neighbors is counted and corrected from edge effects. \\ 
    
    $\widehat{M} \left( r \right) $ 
    & $\displaystyle n \left( x_i, r \right) = \sum_{j, i \ne j}{\mathbf{1} \left( \left\| x_i - x^c_j \right\| \le r \right) w \left( x^c_j \right)}$ 
    & Each neighbor of the type of interest counts for its weight. \\
    
    \bottomrule
  \end{tabularx}
  \label{table:step1}
\end{table*}
 
Un exemple de figure est fourni ici (logo de l'UMR EcoFoG, figure~\ref{fig:ecofog2021})

\begin{figure}
  \centering
  \includegraphics[width=0.8\linewidth]{EcoFoG2021}
  \caption{Logo de l'UMR EcoFoG}
  \label{fig:ecofog2021}
\end{figure}


\section{Texte de remplissage}\label{sec:remplissage}
 
\lipsum

Modèle de liste:
\begin{itemize}
  \item premier point;
  \item deuxième point.
\end{itemize}


%----------------------------------------------------------------------------------------
%	REFERENCE LIST
%----------------------------------------------------------------------------------------

\bibliographystyle{mee}
\bibliography{references}

%----------------------------------------------------------------------------------------

\end{document}
